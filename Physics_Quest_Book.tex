\documentclass[12pt,a4paper]{report}
\usepackage[utf8]{inputenc}
\usepackage[russian]{babel}

\usepackage{geometry}
 \geometry{
 a4paper,
 total={170mm,257mm},
 left=20mm,
 top=20mm,
 }
 
\usepackage[linkcolor=true,linktoc=none]{hyperref}
\usepackage[
    type={CC},
    modifier={by},
    version={4.0},
]{doclicense}

\title{Вопросы по всем понятиям\\из Мякишева Г.Я.\\Физика для 10-11 классов}
\author{Горохов М.А.}
\date{Версия: 1.0 (2 Января 2022)}

\begin{document}

\setcounter{tocdepth}{0}
\maketitle

\tableofcontents

\chapter*{Предисловие}
\addcontentsline{toc}{chapter}{Предисловие}

\section{О книге и об авторе}
Я начинал давать уроки по физике ученикам осенью 2019г., когда я поступил на 1-й курс Прикладной Математики и Информатики МАИ. С учениками 10-11 классов мы шли по 5-ти томнику Мякишева профильного уровня, по которому я сам учился в школьное время. К своим занятиям я составлял контрольные вопросы на каждое понятие из учебника. Я подумал: <<Как же было бы классно, если бы каждое понятие, каждая(!) единица знания была вынесена из учебника в отдельный список в сухом остатке!>>. С помощью эти вопросов я мог легко ориентироваться в занятии, я мог оценить, как ученик понимает тему. Позже я заметил, что у этих вопросов есть широкое применение. И~вот спустя~2~года я решил оформить их в виде книжки-пособия. Некоторые вопросы были составлены в 2019г., некоторые летом 2020г., остальные --- уже в конце декабря 2021г. во время написания этой книги. \\

Применение книги:
\begin{itemize}
\item Позволяет понять структуру физики 10-11 класса
\item Можно использовать в качестве подробного план урока и всего школьного курса физики
\item Можно самостоятельно проверять свое понимание. Как следствие, ты можешь лучше освоить учебник, лучше подготовиться к школьным урокам, контрольным, экзаменам или олимпиадам, 
\item Учитель может использовать вопросы как основу для своей контрольной/самостоятельной, или задавать их ученикам на уроке
\item А ещё книга может быть полезна студентам вузов с непрофильной физикой. Вот только, я думаю, вам нужно будет добавить немного матана, линала, производных с интегралами, и других областей прекрасной высшей математики
\end{itemize}

Если формулировать иначе, эту книгу можно использовать как:
\begin{itemize}
\item Для ученика: подробный конспект 5 томника Мякишева
\item Для ученика: вопросы для самопроверки 
\item Для ученика и учителя: подробный план урока. Ученик понимает структуру уроков, учитель лучше ориентируется в занятиях
\item Для учителя: основа для домашних, самостоятельных и контрольных работ
\item Для учителя: <<блиц>> опрос для школьников на занятиях
\end{itemize}

Это некоторые идеи. После работы с контрольными вопросами мои ученики замечали, как у них появлялось четкое понимание структуры физики 10-11 класса: что входит в физику, из чего состоит, что понимают или не понимают, и что предстоит изучить. Это~основная~задача~пособия: \textbf{\textsl{дать чёткое понимание структуры физики}} \\

Многие вопросы я переформулировал. Например, <<Понятие материальной точки>> я заменил на <<Что такое материальная точка?>>. Или <<Графики. Больше графиков. График проекции (модуля) скорости и ускорения>> на <<Как будут выглядеть графики скорости, пути, координаты при равноускоренном движении?>>. То есть, я перевёл в нормальные человеческие вопросы, повелительные глаголы и рекомендации. Так легче понимать вопросы. \\

В книге могут быть неточности, ошибки, неправильное использование стилей и неполнота материала. Иногда я намеренно использую тавтологию, чтобы читатель лучше понимал ход мыслей. Я буду очень рад, если вы будете мне писать пожелания, замечания, отзывы, и поможете сделать книгу лучше! А ещё вы можете сделать форк книги. Ссылка на мой гитхаб будет ниже. Книга написана на Overleaf с помощью \LaTeX \\

Меня зовут Михаил Горохов, я --- автор оригинала книги-пособия, и на момент написания, я студент 3-го курса ПМИ МАИ. Интересуюсь педагогикой, образованием, математикой, нейросетями, компьютерным зрением (Computer Vision aka CV) и многим другим; стремлюсь делать мир лучше!


\section{Контакты}
ВКонтакте: \href{https://vk.com/magorokhoov}{https://vk.com/magorokhoov} \\
Телеграм \href{https://t.me/magorokhoov}{@magorokhoov} \\
Github: \href{https://github.com/magorokhoov}{https://github.com/magorokhoov}\\
Github книги: \href{https://github.com/magorokhoov/Physics-Quest-Book}{https://github.com/magorokhoov/Physics-Quest-Book}\\


\section{Структура книги. Как использовать книгу?}

Я читал учебник и по ходу чтения я выписывал в отдельный файлик контрольные вопросы. Таким образом, \textbf{\textsl{хронологический порядок был сохранен}}. В~каком~порядке идут вопросы, в таком порядке будут идти ответы в Мякишеве.

Эта книга охватывает все 5 томов профильного уровня. Каждая часть этой книги соответствует тому Мякишева. Каждая глава соответствует главе из Мякишева. Поэтому вы будете легко ориентироваться в контрольных вопросах.

Каждая глава содержит контрольные вопросы. В начале главы я могу дать комментарии. Первые несколько глав оканчиваются некоторыми примерами ответов. Имейте в виду, так как я писал по профильному Мякишеву и старался выписать почти каждую единицу знания, то многие вопросы не будут актуальны для~ЕГЭ, для~олимпиад, для~ваших школьных занятий. Поэтому отбирайте вопросы, исходя из потребностей и интереса. (Ну~правда, вопросов получилось очень~много, более~770, и~я~думаю, многое~знать~необязательно. Это~касается тонких частностей, приборов и специфичных явлений физики)

Наиболее проработанными являются первые 3 тома: Механика, Термодинамика и Электродинамика. 

\paragraph{Некоторые идеи, как пользоваться книгой.}
Эту книгу можно применять по-разному, но основное её применение --- это \textbf{\textsl{приложение к Мякишеву}}.

Предположим, ты ученик 10 класса, и ты хочешь лучше знать физику, без привязки к олимпиадам или экзаменам. Тогда ты можешь изучать Мякишева, а вопросы использовать для самопроверки. Если что-то не понимаешь --- стараешься разобраться, перечитываешь учебник, или спрашиваешь друга, или учителя.

Если ты ученик средней школы, то ты можешь ознакомиться с материалом профильной физики 10-11 класса.

Если вы готовитесь сдавать ЕГЭ, то вы можете понять, какие темы вы знаете, какие не знаете, что следует подтянуть. Я рекомендую свериться с кодификаторами/спецификаторами к ЕГЭ, потому что, как я говорил, многие темы на ЕГЭ требоваться не будут.

Если ты любознательный ученик или будущий олимпиадник, то ты можешь вдохновиться предыдущими примерами, и использовать книгу как захочется и как посчитаешь нужным.

Если вы учитель или репетитор, то вы можете использовать контрольные вопросы, как чёткий план урока. Или как вопросы, которые можно задать ученикам в формате <<близ опроса>>. Вы можете использовать их для написания своих контрольных и самостоятельных работ. Или в качестве домашней работы для учеников.

В общем, применяйте книгу, как посчитаете нужным.

\chapter*{Лицензия}
\addcontentsline{toc}{chapter}{Лицензия}
\doclicenseThis
То есть, \textbf{каждый может}:
\begin{enumerate}
\item Использовать книгу в любых целях (коммерческих и некоммерческих)
\item Делиться (обмениваться) --- копировать и распространять материал на любом носителе и в любом формате 
\item Адаптировать (создавать производные материалы) --- видоизменять, создавать новое, опираясь на этот материал в любых целях, включая коммерческие.
\end{enumerate}
\textbf{Условия}:
\begin{enumerate}
\item Авторство оригинала книги должно быть сохранено
\item Если были сделаны изменения, следует их обозначить. Вы можете это делать любым разумным способом
\end{enumerate}

Исходный~\LaTeX~код книги будет доступен на GitHub под лицензией Apache 2.0

\part{Механика}
\setcounter{chapter}{-1}

\chapter{Введение в механику}
\section{Вопросы}
\begin{enumerate} 
\item Что такое наука? 
\item Что изучает физика? 
\item Сформулируйте две основные цели физики
\item Что такое физический закон? 
\item Что такое теория и практика? Опишите связь между ними
\item Что такое упрощенная модель. Для чего нужна упрощенная модель? 
\item Что такое математического язык и почему физика использует его?
\item Что такое механического движение?
\item Что такое пространство и время в физике? Своими словами, ваше понимание
\item Расскажите вкратце про Исаака Ньютона и его труды
\item Что такое классическая механика? Что включает в себя классическая механика?
\item Что такое механика?
\end{enumerate}

\section{Примеры ответов}
\begin{itemize}
\item Физика --- это наука, занимающаяся изучением самых общих свойств окружающего нас материального мира
\item Две цели физики: 1. найти наиболее общие законы природы 2. объяснить конкретные процессы с помощью этих общих законов
\item Механика --- наука об общих законах движения тел
\end{itemize} 

\chapter{Кинематика}
\section{Основы кинематики}
\begin{enumerate}
\item Что изучает кинематика? 
\item Что такое материальная точка?
\item Что такое координаты?
\item Что такое система отчета (СО)? Как задать систему отсчета?
\item Перечислите основные способы описания движения (минимум 4 способа)
\item Что такое траектория?
\item Что такое равномерное прямолинейное движение?
\item Что такое путь? Единица измерения пути
\item Единица измерения времени
\item Что такое скорость? Формула скорости. Единица измерения
\item Как задать уравнение равномерного прямолинейного движения (равн.прям.движ)?
\item Как будут выглядеть графики скорости, пути, координаты при равн.прям.движ?
\item Что такое средняя скорость? Формула
\item Что такое мгновенная скорость? Формула
\item Что такое вектор? Что такое модуль вектора
\item Что такое перемещение?
\item Что такое радиус-вектор и как он обозначается?
\item Что такое проекция вектора?
\item Что такое скаляр?
\item Какие вы знаете операции с векторами?
\item Что такое путевая скорость?
\item Что такое ускорение? Формула
\item Как задать уравнение равноускоренного движения?
\item Как будут выглядеть графики скорости, пути, координаты при равноускоренном движении?
\item Три формулы перемещения (без конечной скорости; без времени; без ускорения)
\item Самостоятельно выведите три формулы перемещения из предыдущего пункта (в Мякишеве есть выведение) 
\item Как решать задачи по кинематике? Назовите 6 пунктов
\end{enumerate} 

\section{Примеры ответов}
\begin{itemize}
\item Кинематика изучает способы описания движения и связь между величинами, характеризующих эти движения
\item Тело отчета --- тело, относительно которого рассматривается движение
\item Перемещение --- направленный отрезок, проведенный из начального положения движущейся точки в её конечное положение
\item Длину вектора перемещения называют модулем
\item Ускорение --- величина, которая характеризует быстроту изменения скорости. Обозначение $a$ (с англ. acceleration). $a = a_0 + \upsilon t$
\end{itemize} 

\section{Баллистика}
\begin{enumerate}
\item Что такое свободное падение?
\item Расскажите опыт Галилея и его суть
\item Расскажите опыт Ньютона и его суть
\item Расскажите вкратце про Галилео Галилея и его труды
\item Что такое ускорение свободного падения?
\item Какая будет траектория у свободно падающего тела?
\item Приведите формулу зависимости времени подъема и времени полета
\item Как изменяется скорость по горизонтали и по вертикали? Приведите формулы
\item Выразить $V_x(t)$, $V_y(t)$, $x(t)$, $y(t)$, считая известным начальную скорость $V_0$ и угол $\alpha$
\item Что такое баллистическая кривая и чем она отличается от параболы?
\item Выразите наибольшую высоту подъема, зная вектор начальной скорости и угол броска
\item Что такое дальность полета? Формула дальности полета
\end{enumerate}

\section{Примеры ответов}
\begin{itemize}
\item Свободное падение --- движение тело только под влиянием притяжения к планете (в частности, к Земле)
\item Дальность полета --- перемещение брошенного тела по OX (по горизонтали) от точки броска до точки падения. $L = \frac{{\upsilon_0}^2 \sin 2\alpha}{g}$
\end{itemize} 

\section{Движение по окружности. Виды ускорений. Относительность движения}
\begin{enumerate}
\item Что такое равномерное движение по окружности?
\item Что такое центростремительное (нормальное), тангенциальное и полное ускорения? Назовите физический смысл нормального и тангенциального ускорений
\item Приведите примеры нормального ускорения и тангенциального ускорения
\item Что такое радиус кривизны окружности?
\item Что такое угловая скорость? Выведите формулу: а. через изменение угла б. через период
\item Что такое угловое ускорение?
\item Как будут выглядеть функции $\phi(t)$, $\omega(t)$ при равноускоренном движении по окружности?
\item Что такое линейная скорость?
\item Какая связь между линейной скоростью и угловой скоростью?
\item Выразите центростремительное ускорение через угловую скорость
\item Какая связь углового ускорения с линейным ускорением?
\item Что описывают преобразования Галилея? Опишите эти преобразования для координаты, скорости и ускорения.
\item Что такое абсолютная, относительная и переносная скорости? Дайте замечание по поводу определения скорости.
\end{enumerate}

\section{Примеры ответов}
\begin{itemize}
\item Равномерное движение по окружности --- движение с постоянной по модулю скоростью $|\upsilon| = const$ по траектории окружности
\item Связь между угловым ускорением и линейным: $a_\tau = \frac{d\upsilon}{dt} = \beta R$, где $a_\tau$ --- линейное тангенциальное ускорение, $\beta$ --- угловое ускорение
\end{itemize} 

\chapter{Основы динамики. Законы Ньютона}
\section{Законы механики Ньютона}
\begin{enumerate}
\item Что изучает динамика?
\item В результате чего возникает у тел изменение скорости (ускорение)?
\item Что такое свободное тело?
\item Что такое инерциальная и неинерциальная системы? Чем они отличаются?
\item Почему основные законы механики относятся к материальным точкам, а не к произвольным телам?
\item Когда тело можно считать материальной точкой?
\item Сформулируйте первый закон Ньютона
\item Что такое геоцентрическая система отсчёта?
\item Что такое сила? Что характеризует сила?
\item Что такое динамометр?
\item Что такое инерция?
\item Что такое масса?
\item Сформулируйте второй закон Ньютона
\item Сформулируйте третий закон Ньютона. Какое замечание стоит сделать?
\item Дайте краткая справку про международную систему единиц (СИ) и про другие системы
\item В чем измеряется сила в СИ и в других системах?
\item В чем заключается прямая задача механики? В чем заключается обратная задача механики?
\item Что такое механическое состояние?
\item Сформулируйте три элемента, составляющие любую фундаментальную теорию
\item Сформулируйте принцип относительности
\item Сформулируйте алгоритм решения задач на динамику
\end{enumerate}

\section{Больше блоков}
\begin{enumerate} 
\item Что такое неподвижный блок? Какой физический смысл у неподвижного блока?
\item Что такое подвижного блока. Какой физический смысл у подвижного блока?
\item Что означает нерастяжимость, неразрывность и невесомость нити?
\end{enumerate}

\chapter{Силы в Механике}
\section{Вопросы}
\begin{enumerate}
\item Сформулируйте четыре типа сил и расскажите про их особенности
\item В чем суть догадки Ньютона?
\item Дайте краткую биографическую справку про Кеплера
\item Сформулируйте первый закон Кеплера
\item Сформулируйте второй закон Кеплера
\item Сформулируйте третий закон Кеплера
\item Сформулируйте закон всемирного тяготения
\item Что такое центральные силы?
\item Какой физический смысл у гравитационной постоянной?
\item В чем суть опыта Кавендиша?
\item Что такое гравитационная масса и инертная масса?
\item Что такое сила тяжести?
\item Что такое ускорение свободного падения?
\item Что такое первая космическая и вторая космические скорости?
\item Выведите первую и вторую космические скорости
\item Что такое сила упругости? Какая у неё природа?
\item Что такое упругие тела? Что такое пластичные тела? 
\item Что такое нормальная реакция опоры?
\item Сформулируйте закон Гука
\item Что характеризует Коэффициент жесткости (упругости)?
\item Как находится коэффициент жесткости системы пружин соединенных а. параллельно б. последовательно
\item Что такое вес тела? Что такое невесомости? Что такое перегрузка?
\item Сила нормальной реакции опоры приложена к телу или к поверхности? А к чему приложен вес?
\item Что такое сила трения покоя, скольжения и качения?
\item Что такое движение в вязкой среде? Какие формулы вы знаете?
\end{enumerate}

\chapter{Силы инерции}
\section{Вопросы}
\begin{enumerate}
\item Дайте бытовое понятие инерциальной и неинерциальной системам отсчета.
\item Что такое сила инерции? Зачем их вводят?
\item Приведите пример инерциальной и неинерциальной систем.
\item Сравните инерциальную и не инерциальную систему отсчета (на примере тележки с маятником).
\item Расскажите кратко про вращающиеся система отсчета.

\end{enumerate}

\chapter{Импульс. ЗСИ}
\section{Вопросы}
\begin{enumerate}
\item Что делает законы сохранения очень важными?
\item С чем фундаментально связан закон сохранения импульса?
\item А с чем связан закон сохранения энергии?
\item Что такое импульс материальной точки?
\item Что такое импульс силы?
\item Сформулируйте закон сохранения импульса
\item Что такое замкнутая система? Когда выполняется ЗСИ?
\item Что такое реактивное движение и что такое реактивная сила?
\item Напишите уравнение Мещерского
\item Расскажите кратко про реактивные двигатели.
\item Когда был запущен первый искусственный спутник Земли?
\item Дайте краткую справку про Циолковского Константина Эдуардовича
\item Дайте краткую справку про Королева Сергеева Павловича
\end{enumerate}

\chapter{Энергия. ЗСЭ}
\section{Вопросы}
\begin{enumerate} 
\item Какое назначение двигателей?
\item Что такое работа? Напишите формула работы. Какой физический смысл у работы? Единица измерения работы.
\item Какая полная работа у нескольких сил, приложенных к телу?
\item Что такое мощность. Напишите формулу мощности. Какой физический смысл мощности?
\item <<Тело обладает энергией>> - что это означает?
\item Что такое кинетическая энергия? Напишите формулу. Какой физический смысл?
\item Что такое потенциальная энергия? Напишите формулу. Какой физический смысл?
\item Какая взаимосвязь между работой силы тяжести и потенциальной энергией тела?
\item Зависит ли работа силы тяжести от траектории?
\item напишите формулу работы для силы упругости в пружине
\item Как называют силы, зависящие от расстояния, но не зависящие от скорости?
\item Сформулируйте Закон Сохранения Энергии (ЗСЭ). Напишите формулу
\item Чему равно изменение полной механической энергии?
\item Что такое абсолютно упругий удар? Что такое абсолютно неупругий удар?
\item Сила трения --- консервативная или неконсервативная? Объясните свое решение.
\end{enumerate}

\chapter{Движение твердого тела}
\section{Вопросы}
\begin{enumerate}
\item Вспомните из кинематики, что означает <<описать движение>>?
\item Какое тело можно считать абсолютно твердым?
\item Что такое поступательное движение? А что такое вращательное движение? А плоскопараллельное?
\item Что такое мгновенный центр вращения?
\item Что такое центр массы тела? Напишите формулу
\item Чему равен импульс абсолютно твердого тела?
\item Сформулируйте теорему о движении центра масс
\item Что такое момент силы? Физический смысл. Какая связь с обыкновенной силой?
\item Что такое момент инерции? Физический смысл. Какая связь с обыкновенной массой?
\item Что такое момент импульса? Физический смысл. Какая связь с обыкновенным импульсом?
\item Сформулируйте закон Сохранения Момента Импульса
\end{enumerate}

\chapter{Статика}
\section{Вопросы}
\begin{enumerate}
\item Как называется раздел, который изучает равновесие абсолютно твердых тел?
\item Сформулируйте первое условие равновесия (подсказка, про силы)
\item Сформулируйте второе условие равновесия (подсказка, про момент)
\item Напишите уравнение моментов
\item Расскажите про равновесие деформируемых тел
\item Что такое центр тяжести? Какая связь с центром массы? Всегда ли совпадают? А в каком случае они не совпадут? Приведите примеры
\item Что такое устойчивое, неустойчивое и безразличное равновесии?
\item В чем суть принципа минимальной энергии? Какая связь с устойчивым и неустойчивым равновесиями?

\end{enumerate}

\chapter{Механика деформируемых тел. Давление, гидродинамика, гидростатика}
\section{Вопросы}
\begin{enumerate}
\item Сравните твердые тела, жидкости и газы. В чем отличие и сходства между друг другом? Используйте 4 параметра для сравнения
\item Опишите все 5 видов деформации
\item Что такое механическое напряжение? Какой физический смысл. Напишите формулу. Единицы измерения
\item Сформулируйте закон Гука
\item Что такое модуль Юнга?
\item Что означает <<текучесть материала>>?
\item Что такое пластичность?
\item Что такое хрупкость?
\item Что такое давление? Какой физический смысл? Напишите формулу
\item Что такое гидростатическое давление? Какой физический смысл? Напишите формулу
\item Что такое сообщающиеся сосуды?
\item Сформулируйте закон Паскаля
\item Расскажите про гидростатический парадокс
\item Что такое гидравлический пресс? Какой у него принцип работы? Напишите формулу. Где применяется?
\item Сформулируйте закон Архимеда. Какой физический смысл? Напишите формулу
\item Что такое ламинарное течение? Что такое турбулентное течение?
\item Напишите уравнение Бернулли. Зачем оно нужно?
\end{enumerate}


%%%%%%%%%%%%%%%%%%%%%%%%%%%%%%%
%%%%%%%%%%%%%%%%%%%%%%%%%%%%%%%
%%%%%%%%%%%%%%%%%%%%%%%%%%%%%%%



\part{Молекулярная физика. Термодинамика}
\setcounter{chapter}{0}
\chapter{Введение в Термодинамику}
\section{Вопросы}
\begin{enumerate}
\item Что позволяет определить механика Ньютона?
\item От каких латинских слов произошло слово "молекула"?
\item Каковы причины краха механической теории?
\item В чем суть вещественной теории тепла? Расскажите про теорию теплорода
\item Расскажите кратко про М.В. Ломоносова.
\item Каковы причины краха теории теплорода? Чем была заменена эта теория?
\item Что изучаем термодинамика?
\item К каким телам применимы все законы термодинамики?
\item Какая задача ставится перед Молекулярно-Кинетической теорией?

\end{enumerate}

\chapter{Основы Молекулярной-Кинетической Теории (МКТ)}
\section{Вопросы}
\begin{enumerate}
\item Расскажите про атомную гипотезу
\item Дайте оценку диаметру атома углерода. Дайте оценку количества молекул в одной капле воды на 1гр.
\item Что такое относительная молекулярная масса?
\item Что такое "количество вещества"? Напишите формулу. Единица измерения
\item Что такое постоянная Авогадро?
\item Что такое молярная масса?
\item Что такое тепловое движение?
\item Что такое броуновское движение?
\item Что такое электрический диполь?
\item Расскажите про разные состояния тела: газ, жидкость и твердого состояние. Дайте характеристику каждого состояния
\end{enumerate}

\chapter{Температура. Газовые законы}
\section{Вопросы}
\begin{enumerate}
\item Что такое макроскопические параметры системы? Какие величины относят к этим параметрам?
\item Что такое тепловое (термодинамическое) равновесие?
\item Что такое температура?
\item Что такое уравнение состояния?
\item Что такое термодинамический процесс?
\item Что такое равновесность, неравновесность и время релаксации?
\item Что такое газовый закон?
\item Расскажите про закон Бойля-Мариотта. Что остается постоянным? Как называется этот процесс? Дайте формулу. Какое название у графика?
\item Расскажите про закон Гей-Люссака. Что остается постоянным? Как называется этот процесс? Дайте формулу. Какое название у графика?
\item Что такое идеальный газ?
\item Что такое абсолютный нуль? Какое значение по Цельсию и по Кельвину?
\item Расскажите про шкалу Кельвина
\item Расскажите про первоначальную гипотезу Авогадро. Расскажите про строгий закон Авогадро.
\item Расскажите закон Дальтона
\item Что такое уравнение состояния газа? Как по-другому называют это уравнение?
\item Расскажите про закон Шарля. Что остается постоянным? Как называется этот процесс? Дайте формулу. Какое название у графика?
\item Какое вы знаете применением свойствам газа?
\end{enumerate}

\section{Примеры ответов}
\begin{enumerate}
\setcounter{enumi}{14}
\item Уравнение Менделеева-Клапейрона
\end{enumerate}

\chapter{МКТ идеального газа}
Эта глава посвящена выведению законов и уравнений МКТ. Многие пункты я пометил звездочкой <<*>>
\section{Вопросы}
\begin{enumerate}
\item *Как вы понимаете, что такое <<количественная теория процессов>>?
\item *Что позволяют делать законы статистической механики? Почему классическая механика затрудняется в описании количественной теории процессов?
\item *Напишите формулу для <<среднего по времени квадрата скорости>> молекул газа
\item *Напишите формулу для статистически средней скорости молекул газа
\item Что такое физическая модель?
\item Что такое идеальный газ?
\item При каких условиях реальный газ становится подобным идеальному?
\item Напишите основное уравнение МКТ. а. Через скорость б. через кинетическую энергию 
\item *Выведите основное уравнение МКТ
\item Напишите формулу, связывающую температуру и среднюю кинетическую энергию
\item Какой физический смысл у постоянной Больцмана?
\item Напишите уравнение зависимости давления от концентрации и температуры
\item В чем заключается роль быстрых молекул?
\item Напишите формулу для средней скорости броуновской частицы
\item Напишите формулу для внутренней энергии $i$-атомного идеального газа
\end{enumerate}

\chapter{Законы термодинамики}
\section{Вопросы}
\begin{enumerate}
\item Вспомните роль работы в механике
\item Расскажите про геометрическое понимание работы (на графике $pV$)
\item Что такое калориметр?
\item Напишите уравнение теплого баланса: а. для $2$-х тел, б. для $n$-тел 
\item Напишите уравнение количества теплоты тела
\item Что такое теплоемкость? Чем отличается молярная теплоемкость от массовой?
\item Что такое калория? Чему равна калория в СИ?
\item Расскажите про опыт Джоуля
\item Расскажите кратко про Джоуля Джеймса Прескотт
\item Какой очень важный вывод об энергии был сделан в результате многочисленных опытов?
\item *Чему равна внутренняя энергия макроскопического тела с точки зрения МКТ?
\item Расскажите про первый закон термодинамики. Напишите формулу. В чем суть этого закона?
\item Чему равна теплоемкость газа при постоянном: а. объеме б. давлении
\item Что обозначает величина $R$?
\item Как называют процесс в теплоизолированной системе?
\item Что означает понятие <<необратимый процесс>>?
\item Расскажите про второй закон термодинамики. Дайте определение: а. по Р. Клаузиусу б. по У.Кельвину 
\item Что такое КПД? Напишите формулу для тепловой машины
\item Расскажите про мысленный эксперимент <<Демон Максвелла>>
\item Что такое тепловой двигатель?
\item Что такое нагреватель в тепловом двигателе и какова его роль?
\item Что такое холодильник в тепловом двигателе и какова его роль?
\item Расскажите как работает тепловой двигатель
\item Расскажите про применение тепловых двигателей
\item Расскажите про идеальную машину Карно. Что позволяет оценить машина Карно?
\item Напишите формулу для КПД для машины Карно
\item Расскажите про Карно Никола Леонара Сади
\item Чем отличается тепловая машина от холодильной?
\item Что такое тепловой насос? Расскажите принцип его работы и его применение
\item Что такое адиобатный, изохороный, изотермический, изобарический процессы?
\item Что такое адиобата, изохора, изотерма, изобара?
\item Какие законы термодинамики вы знаете?
\end{enumerate}

\chapter{Взаимные превращения жидкостей и газов}

\section{Вопросы}
\begin{enumerate}
\item Что такое испарение?
\item Расскажите про испарение с точки зрения МКТ
\item Что такое конденсация?
\item Почему возникает охлаждение при испарении?
\item Что такое сублимация (возгонка)?
\item Что такое насыщенным пар? Что такое ненасыщенный пар?
\item Какое равновесие называется динамичным (подвижным)?
\item Что называют давлением насыщенного пара?
\item Какая зависимость давления и плотности насыщенного пара от температуры?
\item Что такое критическая температура для воды?
\item Что такое кипение? Что происходит при кипении?
\item Что такое удельная теплота парообразования? Напишите формулу
\item *Что такое детандер?
\item Расскажите про Капицу Петра Леонидовича
\item Что такое парциальное давление водяного пара?
\item Что такое абсолютная и относительная влажности?
\item Что такое точка росы?
\item Что такое гигрометр?
\item Что такое психрометр?
\end{enumerate}

\chapter{Поверхностное натяжение в жидкостях}
\section{Вопросы}
\begin{enumerate}
\item Каким удивительным явлением обладает поверхность жидкости?
\item Что такое поверхностная энергия?
\item Что такое удельная поверхностная энергия?
\item Что такое поверхностная сила?
\item Что такое мениск?
\item Что такое капиллярное явление?
\item Напишите формулу для определения поверхностного натяжения через плотность жидкости
\end{enumerate}

\chapter{Твердые тела и их превращения в жидкости}

\section{Вопросы}
\begin{enumerate}
\item Что такое кристалл?
\item Что такое монокристалл? Что такое поликристалл?
\item Что такое полиморфизм?
\item Что такое анизотропия?
\item Что такое кристаллическая решетка? Что такое период кристаллической решетки?
\item Какие вы знаете типы кристаллов? Назовите 4 типа
\item Приведите пример ковалентных кристаллов
\item Что такое аморфное тело? Чем отличается от кристаллического тела?
\item Какие вы знаете типы жидких кристаллов? Назовите 3 типа. Расскажите про каждый тип
\item Расскажите про применение жидких кристаллов
\item Что называют дефектами в кристаллах?
\item Назовите два вида диффузии в твердых телах по версии Френкеля
\item Что такое дислокация в кристаллах?
\item Как связаны дислокация и прочность материала?
\item Что называют температурой кристаллизации?
\item Что называют переохлаждением жидкости?
\item Что такое удельная теплота плавления?
\item Какие вы знаете зависимости температуры плавления от давления?
\item Что такое тройная точка?
\end{enumerate}

\chapter{Тепловое расширение твердых и жидких тел}
\section{Вопросы}
\begin{enumerate}
\item Что такое тепловое расширение?
\item Что называют температурным коэффициентом линейного расширения?
\item Расскажите про особенности расширения воды
\item Как учитывают тепловое расширение тел?
\item Как используют тепловое расширение в технике?
\end{enumerate}


%%%%%%%%%%%%%%%%%%%%%%%%%%%%%%%
%%%%%%%%%%%%%%%%%%%%%%%%%%%%%%%
%%%%%%%%%%%%%%%%%%%%%%%%%%%%%%%

\part{Электродинамика}
\setcounter{chapter}{-1}
\chapter{Введение в электродинамику}
\section{Вопросы}
\begin{enumerate}
\item Вспомните 4 типа сил. Какие имеют наибольшую интенсивность? Какая у них сфера действия? Расскажите каждую силу
\item Расскажите кратко про Максвелла Джеймса Клерка
\item Приведите примеры, в основу каких явлений заложены законы электродинамики
\item Что определяет электрический заряд?
\item Чем отличаются знаки заряда?
\item Что такое элементарный заряд?
\item Расскажите немного про кварки.
\item Сформулируйте закон сохранения заряда для замкнутой системы
\item Почему трудно определить понятие <<электрический заряд>>?
\end{enumerate}


\chapter{Электростатика}
\section{Заряженные тела}
\begin{enumerate}
\item Что изучает электростатика?
\item Что такое электрометр?
\item Опишите простой способ получения электрически заряженного тела с помощью волос и расчески
\item Что такое электризация?
\item Как происходит электризация тел?
\item Приведите примеры применения явления электризации в технике
\item Чем обусловлены знак заряда заряженного макроскопического тела?
\item Расскажите про Кулона Шарля Огюстена
\item В чем заключается опыт Кулона?
\item Расскажите закон Кулона. На какой закон похож закон Кулона? Напишите формулу закона Кулона в векторной форме
\item Расскажите про единицу измерения электрического заряда в абсолютной (Гауссовской) системе и в СИ. В чем отличие зарядов и этих системах?
\item Чему равен заряд электрона?
\item Что такое диэлектрическая проницаемость? Напишите формулу
\item Вспомните про модуль Юнга. Что определяет модуль Юнга?
\end{enumerate}


\section{Поля. Напряженность Поля. Теорема Гаусса. Диэлектрики}
\begin{enumerate}
\item В чем идея теории близкодействия?
\item В чем идея теории дальнодействия?
\item Расскажите про идею Майкла Фарадея об воздействии электрических зарядов друг на друга. Посредством чего, по идеи Фарадея, заряды воздействуют друг на друга?
\item Расскажите про Майла Фарадея
\item В чем заключается главное отличие теории дальнодействия от теории близкодействия?
\item Что такое электрическое поле? Какими свойствами оно обладает?
\item Какое вы знаете главное свойство электрического поля?
\item Как называют электрическое поле неподвижных зарядов? Что создает электростатическое поле?
\item Что такое напряженность электростатического поля? Напишите скалярную и векторную формулы
\item Как называется принцип, благодаря которому можно найти напряженность в данной точке, геометрически просуммировав напряженности, создаваемыми различными электрическими полями в данной точке?
\item Что такое линия напряженности?  напишите формулу нахождения количества линий напряженности
\item Что такое поток напряженности электрического поля?
\item Что такое телесный угол?
\item Расскажите теорему Гаусса о полях. В чем заключается главная идея теормы Гаусса?
\item Расскажите про Карла Фридриха Гаусса
\item Что такое поверхностная плотность заряда?
\item Что такое однородное и неоднородное поля?
\item Какое образуется поле вокруг пластины? А вокруг равномерно заряженной сферы?
\item Что такое объемная плотность заряда?
\item Что позволяет сделать Теорема Гаусса с электрическим полем?
\item Что такое свободный заряд?
\item Что такое среднее значение напряженности?
\item Расскажите про явление электростатической индукции (о распределении заряда в шаре).
\item Где сосредоточен весь заряд у проводника?
\item Что такое диэлектрик? Что создает диэлектрик, находящийся в электростатическом поле? Каким остается диэлектрик?
\item Что такое диэлектрический диполь?
\item Какие вы знаете два вида диэлектрика? Чем они отличаются?
\item Расскажите про явление поляризации полярных диэлектриков
\item Расскажите про явление поляризации неполярных диэлектриков
\item Что такое поляризуемость диэлектрика?
\item Что такое сегнетоэлектрики? Где их применяют?
\item Что такое заряд-изображение? При каких условиях он возникает? (Задача 4. Страница 82 в Мякишеве, Том III. 5 изд.)
\end{enumerate}


\section{Потенциальная энергия поля. Потенциал Поля. Разность потенциалов.}
\begin{enumerate}
\item Что такое консервативные силы? Как ещё называют такие силы?
\item Что означает потенциальность электрических сил? 
\item Как называют энергию электрического поля?
\item Как определяется потенциальная энергия заряда через напряженность?
\item Что такое нулевой уровень потенциальной энергии?
\item Напишите формулу энергии взаимодействия точечных зарядов
\item Чему равна потенциальная энергия системы точечных зарядов?
\item Что такое потенциал? Какой физический смысл у потенциала?
\item Приведите аналогию к потенциалу из Механики
\item Какой потенциал у однородного поля? А у точечного заряда?  А у поля произвольной системы зарядов? Напишите формулы
\item Что такое разность потенциалов и как её ещё называют?
\item В чем измеряется электрический потенциал?
\item Почему потенциал является энергетической характеристикой электрического поля?
\item Какая связь между напряженностью и потенциалом?
\item Как направлена напряженность по отношению к потенциалу?
\item В чем измеряется напряженность?
\item Как называются поверхности равного потенциала?
\item Какой является поверхность проводника? Какими являются все точки проводника?
\item Какие две характеристики имеет электростатическое поле?
\item Какие преимущества разности потенциалов перед напряженностью в отношении характеристики поля?
\item Как называется прибор для измерения разности потенциала? Кратко опишите принцип работы этого прибора
\item Опишите экспериментальное определение элементарного электрического заряда.
\item Придумайте условие простой задачи на потенциалы
\end{enumerate}

\section{Электрическая ёмкость. Конденсаторы.}
\begin{enumerate}
\item Что такое электрическая ёмкость? В чем она измеряется?
\item Что такое уединенный проводник?
\item Как определяется электрическая ёмкость?
\item Напишите формулу для электроемкости шара
\item Какую важную характеристику проводника вы знаете?
\item Что такое конденсатор? Какую систему называют конденсатором?
\item Что понимают под зарядом конденсатора?
\item Напишите формулу для емкости конденсатора через заряд и напряжение. (Вспомните мнемоправило "ку-ку").
\item Как измерить диэлектрическую проницаемость через электроемкости?
\item Какая емкость сферического конденсатора?
\item Назовите различные типы конденсаторов и расскажите про них
\item Какая электроемкость системы при параллельном подключении конденсаторов?
\item Какая электроемкость системы при последовательном подключении конденсаторов?
\item Напишите формулу для энергии плоского конденсатора
\item Приведите примеры применения конденсаторов
\item Пластины заряженного конденсатора попеременно заземляются. Будет ли при этом конденсатор разряжаться?
\end{enumerate}

\chapter{Постоянный электрический ток}
\section{Вопросы}
\begin{enumerate}
\item Что такое электрический ток?
\item Когда и только когда существует электрический ток?
\item Что такое упорядоченное движение заряженных частиц?
\item Какие явления называют действиями электрического тока? Назовите три действия и опишите их
\item Что называют вектором плотности тока?
\item Что такое сила тока?
\item Когда электрический ток называют постоянным?
\item в чем измеряется сила тока в СИ и в абсолютной системе?
\item Назовите два условия возникновения и поддержания электрического тока
\item Что называют источником тока (генератором)?
\item Какое поле называют стационарным?
\item Что такое ВАХ (Вольт-Амперная Характеристика)? Что она показывает?
\item Расскажите про Ома Георга Сиона
\item Что такое проводимость проводника?
\item Что такое электрическое сопротивление?
\item Как связаны проводимость и сопротивление?
\item Расскажите закон Ома. Напишите формулу
\item В чем измеряется сопротивление?
\item Что такое резистор?
\item Что такое удельное сопротивление?
\item Что такое сверхпроводимость?
\item Расскажите про опыты Камерлинга-Оннеса
\item Расскажите применение сверхпроводящих магнитов
\item Что такое работа тока? Напишите формулу
\item Что такое мощность тока? Напишите формулу
\item Расскажите формулировку закона Джоуля-Ленца
\item Какие точки называют разветвлениями (узлами)?
\item Сформулируйте первое правило Кирхгофа
\item Как можно соединять проводники тока?
\item Что такое последовательное соединение проводников?
\item Что такое параллельное соединение проводников?
\item Напишите формулы сопротивления для этих двух типов соединений
\item Что такое амперметр? Какое сопротивление у идеального амперметра?
\item Что такое шунт?
\item Что такое добавочное сопротивление в вольтметре?
\item Как подключают амперметр в электросеть? Как подключают вольтметр в электросеть? 
\item Что называют сторонней силой?
\item Что такое электродвижущая сила (ЭДС)? Напишите формулу
\item Что такое гальванический элемент?
\item Расскажите про опыт Гальвани и про открытие Вольта
\item Что такое элемент Даниэля?
\item Что такое поляризация элемента?
\item Что такое аккумулятор?
\item Опишите устройство кислотного (свинцового) аккумулятора
\item Расскажите закон Ома для полной цепи
\item Расскажите закон ома для участка цепи с ЭДС
\item При каких условиях возникает разрядка аккумулятора? А при каких зарядка?
\item Вспомните первое правило Кирхгофа
\item Сформулируйте второе правило Кирхгофа
\end{enumerate}

\chapter{Электрический ток в различных средах}
\section{Вопросы}
\begin{enumerate}
\item Что такое проводник?
\item Что такое изолятор (диэлектрик)?
\item Что такое пробивная напряженность?
\item Что такое полупроводник?
\item Опишите опыт Рикке
\item Расскажите про Мандельштама Леонида Исааковича
\item Какой вывод можно сделать из закона Ома?
\item Что такое время релаксации?
\item Где находится граница применения закона Ома?
\item Что такое электролитическая диссоциация?
\item Что такое рекомбинирование?
\item Что такое анионы и катионы?
\item Что такое анод и катод?
\item Какую проводимость называют ионной?
\item Опишите закон электролиза
\item Опишите закон Фарадея
\item Что такое постоянная Фарадея?
\item Расскажите про техническое применение электролиза
\item Что такое ионизация газа?
\item Что такое несамостоятельный разряд?
\item Что такое самостоятельный разряд?
\item Что такое ионизация электронным ударом? Что такое электронная лавина?
\item Что такое термоэлектронная эмиссия?
\item Что такое коронным разряд? Что такое искровой разряд? Приведите примеры
\item Что такое плазма? Где находится? Как применяют плазму?
\item Что такое диод? Какая ВАХ у диода?
\item Что такое анодное напряжение? Что такое анодный ток?
\item Что такое триод? Как он устроен?
\item как устроен полупроводник?
\item Расскажите про примесную электропроводимость полупроводников
\item Что такое электронно-дырочный переход?
\item Что такое полупроводниковый диод?
\item Что такое транзистор? Как устроен транзистор?
\item Что такое термистор?
\end{enumerate}

\chapter{Магнитное поле токов}
\section{Вопросы}
\begin{enumerate}
\item В чем заключается открытие Эрстеда?
\item Расскажите про Ампера Андре Мари
\item Что такое магнитное поле?
\item Какие есть основные свойства у магнитного поля?
\item Как называют величину, характеризующую магнитное поле?
\item Опишите принцип суперпозиции для магнитного поля
\item Что такое линии магнитной индукции?
\item Что такое вихревое поле?
\item Что такое магнитный поток? Напишите формулу
\item Опишите закон Био-Савара-Лапласа. напишите формулу
\item Что такое векторное произведение? Как оно обозначается?
\item Расскажите про опыты Ампера
\item Расскажите про закон Ампера
\item Как взаимодействуют параллельные токи?
\item Что называют электродинамической постоянной?
\item Что такое сила Лоренца? Напишите формулу
\item Расскажите про Лоренца Хендрика Антона
\item Совершает ли сила Лоренца работу? Почему?
\item Что такое масс-спектрограф?
\end{enumerate}

\chapter{Электромагнитная индукция}
\section{Вопросы}
\begin{enumerate}
\item В чем заключает открытие Фарадея?
\item Расскажите правило Ленца
\item Что такое электромагнитная индукция?
\item Расскажите закон электромагнитной индукции. Напишите формулу для ЭДС индукции
\item Что такое самоиндукция?
\item Напишите формулу для ЭДС самоиндукции
\item Что такое индуктивность?
\item Что такое взаимная индукция?
\item Напишите формулу для энергии магнитного поля тока
\end{enumerate}

\chapter{Магнитные свойства вещества}
\section{Вопросы}
\begin{enumerate}
\item Что такое магнитная проницаемость?
\item Какие три класса магнитных веществ вы знаете?
\item Что такое остаточный магнетизм?
\item Как объясняется пара- и диамагнетизм?
\item Что такое точка Кюри?
\item Что такое фазовый переход второго рода?
\item Что такое намагниченность?
\item Что такое магнитный гистерезис? Как выглядит график?
\item Что такое ферриты?
\item Что такое домен в магнетизме?
\item Где можно применить ферромагнетики?
\item Где можно применить ферриты?
\end{enumerate}



\part{Колебания и волны}
\setcounter{chapter}{-1}

\chapter{Введение в колебания и волны}
\section{Вопросы}
\begin{enumerate}
\item Где встречаются колебания и волны?
\item Что замечательного в колебаниях и волнах с точки зрения физики?
\item В каких системах всегда возникают колебания?
\end{enumerate}

\chapter{Механические колебания}
\section{Вопросы}
\begin{enumerate}
\item Какие три вида колебаний вы знаете?
\item Расскажите про свободные колебания, про вынужденные колебания, про автоколебания
\item Что такое маятник?
\item Напишите формулу для уравнения колебаний груза на пружине
\item Что такое математический маятник? Что такое физический маятник? Чем они отличаются?
\item Какому замечательному выводу можно прийти, изучив колебания пружины и маятника?
\item По какому закону меняется координата колеблющегося тела?
\item Что такое гармонические колебания?
\item Что такое амплитуда?
\item Что такое период?
\item Что такое частота?
\item Напишите формулу для частоты колебания пружины через её свойства
\item Что такое фаза?
\item Напишите формулу для полной механической энергии пружины с грузом
\item Что такое резонанс?
\item Как применяют резонанс?
\item Как борются с резонансом?
\item Что такое векторная диаграмма?
\item Что такое частотный спектр?
\item Что такое анкер?
\end{enumerate}

\chapter{Электрические колебания}
\section{Вопросы}
\begin{enumerate}
\item Расскажите кратко историю электрических колебаний
\item Что такое электронный осциллограф?
\item Сравните с качественной точки зрения механические колебания и электрические составьте таблицу на 8 пар величин
\item Напишите формулу Томсона
\item Что такое квазистационарный ток? 
\item Что такое действующее значение тока? Что такое действующее значение напряжения? Чему они равны в цепи переменного тока? Напишите формулы
\item Что такое активное сопротивление? 
\item Какой ток пропускает конденсатор, а какой не пропускает?
\item Что такое емкостное сопротивление? 
\item Что такое индуктивное сопротивление? 
\item Напишите формулу для мощности в цепи переменного тока
\item Что такое ламповый генератор? Как он работает?
\item Что такое транзисторный генератор? Как он работает?
\end{enumerate}

\chapter{Логистика и использование электроэнергии}
\section{Вопросы}
\begin{enumerate}
\item Какими преимуществами обладает электроэнергия перед другими видами энергий?
\item Что такое генератор?
\item Что такое индуктор?
\item Что такое якорь в генераторе?
\item Что такое статор?
\item Что такое ротор?
\item Расскажите устройство промышленного генератора
\item Как работает многополосный генератор переменного тока?
\item Что такое трансформатор?
\item Что такое холостой ход трансформатор?
\item Расскажите устройство трансформатора
\item Напишите КПД трансформатора
\item Что такое выпрямитель?
\item Какой ток называют пульсирующим?
\item Как утроен простейший однополупериодный выпрямитель?
\item Как утроен двухполупериодный выпрямитель?
\item Что такое сглаживающий фильтр? Что он делает? Как можно добавить сглаживающий фильтр в выпрямитель из предыдущего пункта?
\item Изобразите график силы тока от времени $i(t)$ в двухполупериодном выпрямителе со сглаживающим фильтром и без
\item Что такое трехфазный ток?
\item Расскажите устройство трехфазного генератора
\item Какие вы знаете соединения обмоток генератора трехфазного тока?
\item Что такое фазное напряжение?
\item Что такое линейное напряжение?
\item Что такое асинхронный электродвигатель?
\item Расскажите про производство и использование электроэнергии
\item Для уменьшения потерь на выделение теплоты в ЛЭП повышают напряжение, уменьшая силу тока. Не противоречит ли это формуле $Q = \frac{U^2}{R}t$?
\end{enumerate}

\chapter{Механические волны. Звук}
\section{Вопросы}
\begin{enumerate}
\item Что такое волна?
\item Что такое поперечные волны? Что такое продольные волны? Чем они отличаются?
\item Что такое длина волны?
\item Напишите формулу скорости распространения волны
\item Что такое бегущая волна? Напишите формулу для бегущей волны 
\item Что называют пучностью в волнах?
\item Что такое стоячая волна? Что такое свободные колебания?
\item Что такое основной тон? Что такое обертон?
\item Какие колебания называются акустическими?
\item Что такое громкость звука? В чем её измеряют?
\item Что такое высота звука? В чем её измеряют?
\item Что такое тембр?
\item Что такое интерференция волн?
\item Что такое когерентные волны?
\item В чем заключается принцип Гюйгенса?
\item Что такое дифракция волн?
\end{enumerate}

\chapter{Электромагнитные волны}
\section{Вопросы}
\begin{enumerate}
\item Что порождает изменяющееся электрическое поле?
\item Какое вы знаете главное условия излучения электромагнитных волн?
\item Какой является электромагнитная волна: поперечной или продольной?
\item Что такое вибратор Герца?
\item Расскажите суть опыта Герца
\item Расскажите про классическую теорию излучения
\item Что такое плотность энергии излучения?
\item Что такое плотность потока излучения? Напишите формулу
\item Какие вы знаете свойства электромагнитных волн?
\item Расскажите про изобретение радио А.С.Попова
\item Расскажите про А.С.Попова
\item Как работает радио?
\item Что такое амплитудная модуляция?
\item Расскажите про другие виды модуляций
\item Что такое радиолокация? 
\end{enumerate}

\part{Оптика. Квантовая физика}
\setcounter{chapter}{0}

\chapter{Геометрическая оптика}
\section{Основы света}
\begin{enumerate}
\item Какие два способа передачи воздействия выделяет Мякишев?
\item Расскажите про корпускулярную и волновую теории света
\item Что такое световой луч?
\item Что изучает геометрическая оптика?
\item Назовите 4 основных закона геометрической оптики
\item Что такое камера-обскура?
\item Что такое поток излучения? Напишите формулу. В чем измеряется?
\item Что такое относительная спектральная световая эффективность?
\item Что такое световой поток?
\item Что такое сила света? Напишите формулу. В чем она измеряется?
\item Что такое стерадиан?
\item Что такое освещенность? Напишите формулу. В чем она измеряется?
\item Сформулируйте закон освещенности
\item Что такое угол падения?
\item Что такое яркость? Напишите формулу. В чем она измеряется?
\item Что такое фотометр? Расскажите про фотометры

\section{Зеркала. Основы оптики}
\item Сформулируйте принцип Ферма
\item Сформулируйте закон отражения
\item Сформулируйте закон преломления. Напишите формулу
\item Какое отражение называется зеркальным (правильным)?
\item Какую поверхность называют зеркальной?
\item Что такое плоское зеркало?
\item Какое изображение называют мнимым?
\item Что такое сферическое зеркало? Опишите его строение
\item Что такое главная оптическая ось? Что такое побочная оптическая ось?
\item Чем отличаются вогнутые и выпуклые зеркала?
\item Напишите формулу сферического зеркала
\item Что такое фокус? Что такое фокусное расстояние? Напишите формулу для сферического зеркала
\item Что такое фокальная плоскость?
\item Что такое мнимым фокус?
\item Что такое оптическая сила? В чем она измеряется?
\item Что такое увеличение? Напишите формулу
\item Что такое абсолютный показатель преломления? Напишите формулу
\item Какую среду называют оптически более плотной?
\item Что такое рефракция?
\item Что такое мираж? Почему он возникает?
\item Что такое полное отражение? Когда оно возникает? Напишите формулу предельного угла

\section{Линзы}
\item Что такое линза?
\item Что такое тонкая линза?
\item Напишите формулу линзы
\item Где находятся фокус и фокальная плоскость для линзы?
\item Напишите формулу для оптической силы линзы
\item Расскажите про правило знаком при использовании формулы тонкой линзы
\item Что такое сферическая аберрация?
\item Что такое хроматическая аберрация?
\item Что такое астигматизм?
\item Что такое фотоаппарат? Опишите принцип его работы
\item Что такое эпидиаскоп?
\item Что такое аккомодация?
\item Что такое дальняя точка аккомодации? Что такое ближняя точка аккомодации?
\item Что такое близорукость? Что такое дальнозоркость?
\item Что такое микроскоп? Как он устроен?
\item Как устроена труба Кеплера?
\item Что такое телескоп?
\item Чему равна оптическая сила системы тонких линз, сложенных вместе?
\end{enumerate}

\chapter{Световые явления}
\section{Вопросы}
\begin{enumerate}
\item Что такое дисперсия света?
\item Какие световые волны называют монохроматическими?
\item Какие волны называют когерентными? 
\item Расскажите про идею Огюстена Френеля, его Бипризму, про идеи Томаса Юнга, про кольца Ньютона, про опыт Юнга
\item Что называют дифракцией Френеля?
\item Что называют дифракцией Фраунгофера?
\item Что такое дифракционная решетка?
\item Какую световую волну называют естественной?
\item Какой свет называют поляризованным?
\end{enumerate}


\chapter{Основы теории относительности}
\section{Вопросы}
\begin{enumerate}
\item Как возникла теория относительности? С какими затруднениями столкнулась Механика Ньютона?
\item Какими тремя возможностями можно было устранить противоречие по мнению Мякишева?
\item Что такое постулат?
\item Какие два постулата лежат в основе теории относительности?
\item Сформулируйте принцип относительности
\item Сформулируйте второй поступал о скорости света
\item Какие события называют одновременными?
\item Напишите формулы преобразования Лоренца
\item Какие 4 замечания можно сделать о преобразованиях Лоренца?
\item Напишите формулу длины движущегося стержня
\item Что такое собственное время?
\item Что такое замедление времени?
\item Для каких систем преобразования Лоренца справедливы?
\item Напишите формулу для сложения скоростей согласно теории относительности
\item Опишите опыт Физо
\item Напишите формулу для релятивистского импульса
\item Напишите релятивистское уравнение движения
\item Сформулируйте принцип соответствия
\item Что такое синхрофазотрон?
\item Напишите формулу приращения массы энергии при увеличении скорости
\item Напишите формулу Эйнштейна
\end{enumerate}

\chapter{Излучения и спектры}
\section{Вопросы}
\begin{enumerate}
\item Что такое тепловое излучение?
\item Что такое электролюминесценция?
\item Что такое катодолюминесценция?
\item Что такое хемилюминесценция?
\item Что такое фотолюминесценция?
\item Что такое спектральная плотность интенсивности излучения?
\item Что такое непрерывный спектр?
\item Что такое линейчатый спектр?
\item Что такое спектральный анализ?
\item Что такое инфракрасное излучение?
\item Что такое ультрафиолетовое излучение?
\item Как открыли рентгеновские лучи?
\item Какие свойства рентгеновских лучей вы знаете?
\item Как можно применить рентгеновские лучи?
\item Расскажите про шкалу электромагнитных излучений
\end{enumerate}

\chapter{Световые кванты. Действия света}
\section{Вопросы}
\begin{enumerate}
\item Что такое <<ультрафиолетовая катастрофа>>? Что произошло?
\item Сформулируйте гипотезу Планка
\item Что такое фотоэффект?
\item Напишите формулу для энергии порции излучения
\item Напишите формулу для работы выхода
\item Что такое красная граница фотоэффекта?
\item Что такое фотон?
\item Напишите формулу для импульса фотона
\item Расскажите про эффект Комптона
\item Что такое волновой вектор?
\item Расскажите про корпускулярно-волновой дуализм
\item Где и как можно применить фотоэффект?
\item Как утроено фотореле?
\end{enumerate}

\chapter{Атомная физика. Квантовая теория}
\section{Вопросы}
\begin{enumerate}
\item Как утроен атом? Расскажите про модель Томсона
\item Расскажите про опыты Резерфорда
\item Расскажите про планетарную модель атома
\item Какие постулаты сформулировал Бор?
\item Напишите формулу для энергии излученного фотона через разность энергий стационарных состояний
\item С какими трудностями столкнулась теория Бора?
\item Расскажите историческую справку про квантовую механику
\item Расскажите про корпускулярно-волновой дуализм
\item Какую гипотезу выдвинул Луи де Бройля?
\item Расскажите про дифракцию и интерференцию электронов
\item Расскажите про мысленные эксперименты, основанные на соотношении неопределенностей
\item Расскажите про интерференцию вероятностей
\item Расскажите про влияние наблюдателя
\item Сформулируйте принцип Паули
\item Что такое лазер?
\item Какими свойствами обладает лазерное излучение? Приведите 3 свойства
\item Расскажите про применение лазеров
\item Расскажите про нелинейную оптику
\end{enumerate}

\chapter{Физика атомного ядра}
\section{Вопросы}
\begin{enumerate}
\item Расскажите про методы наблюдения и регистрации элементарных частиц
\item Что такое радиоактивность?
\item Что такое альфа-частицы? Что такое бета-лучи? Что такое гамма-лучи?
\item Что такое распад атома?
\item Что такое период полураспада?
\item Что такое изотоп?
\item Сформулируйте правила смещения, про альфа распад и бета распад
\item Сформулируйте основные свойства ядерных сил
\item Расскажите про мезоны
\item Что такое энергетический выход ядерной реакции?
\item Расскажите механизм деления ядра урана-235
\item Что такое цепная реакция?
\item Что такое коэффициент размножения нейтронов?
\item Что такое термоядерные реакции?
\item Что такое доза излучения?
\end{enumerate}

\chapter{Элементарные частицы}
\section{Вопросы}
\begin{enumerate}
\item Расскажите про три этапа в развитии физики элементарных частиц
\item Расскажите про распад нейтрона
\item Расскажите про открытие нейтрино
\item Что такое кварки?
\item Что такое глюоны?
\item Расскажите про <<Великое объединение>>
\end{enumerate}

\newpage

The End!

\end{document}